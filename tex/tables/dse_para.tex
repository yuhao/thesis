%!TEX root=../../paper.tex

\begin{table}[p]
\large
\centering
\captionsetup{width=.9\columnwidth}
\caption{Microarchitecture design-space parameters. The first column shows the parameters that are considered in our DSE. The second column shows the metric that the value of each parameter is measured. The $i$::$j$::$k$ in the third column denotes values ranging from $i$ to $k$ at steps of $j$}
\renewcommand*{\arraystretch}{1.4}
\renewcommand*{\tabcolsep}{25pt}
\resizebox{.9\columnwidth}{!}
{
	\begin{tabular}{l l l l l l}
	\toprule[0.15em]
		\bigstrut\textbf{Parameters} & \bigstrut\textbf{Measure} & \bigstrut\textbf{Range}\\
	\midrule[0.05em]
		Issue width				&	count					&	1::1::4	\\
		\# Functional units		&	count					&	1::1::4 \\
		%I-TLB size				&	\#entries				&	8::8::32 \\
		%D-TLB size				&	\#entries				&	8::8::32 \\
		Load queue size			&	\# entries				&	4::4::16 \\
		Store queue size		&	\# entries				&	4::4::16 \\
        Branch prediction size  &   $log_{2}$(\#entries)    &   1::1::10\\
		ROB size				&	\# entries				&	8::8::128 \\
		\# Physical registers	&	\# entries				&	5::5::140 \\
		L1 I-cache size			&	$log_{2}$(KB)			&	3::1::7 \\
		L1 I-cache delay		&	cycles					&	1::1::3 \\
		L1 D-cache size			&	$log_{2}$(KB)			&	3::1::7 \\
		L1 D-cache delay		&	cycles					&	1::1::3\\
		L2 cache size			&	$log_{2}$(KB)			&	7::1::10 \\
		L2 cache delay			&	cycles					&	16,32,64 \\
	\bottomrule[0.15em]
	\end{tabular}
}
\label{tab:dse:para}
\end{table}
