%!TEX root=paper.tex

\section{Retrospective and Prospective Remarks}
\label{sec:conc}

\paragraph{Retrospective} As a promising first step, my proposal explores the feasibility of a holistic system design to improve the energy-efficiency of mobile Web computing while delivering user satisfaction. It argues that the traditional interfaces across the Web computing stack should be enhanced with new abstractions for QoS and hardware. It demonstrates three general principles:

\begin{itemize}
  \item \textsc{Empowering Developers~~} Web developers today must be conscious of energy-efficiency because of increasing user awareness. Current application/runtime abstractions, however, do not provide developers opportunities to optimize for energy-efficiency. Pure runtime-based techniques are QoS-agnostic. A key principle in my proposal is that developers should be integrated into the energy optimization loop by empowering them to express user QoS expectations at an abstract level. \greenweb demonstrates the first step.
  
  \item \textsc{Exposing Hardware Complexity~~} Mobile system-on-chip (SoC) designs are undergoing rapid design iteration and innovation with each generation incorporating more complex cores and accelerators. Future Web browser runtimes must embrace the unprecedented hardware complexity. To enable that, traditional runtime/architecture interfaces must be enhanced to expose a certain level of hardware details. \webrt demonstrates the potential of exposing CPU core type and frequency and designing the Web runtime accordingly.
  
  \item \textsc{General-Purpose vs. Specialization~~} Ultimately future mobile processor designs have to improve both the performance and energy-efficiency simultaneously, not just making trade-offs between the two design goals. Architectural specialization comes as a first choice to achieve both improvements. However, I argue that retaining the general-purpose programmability during the specialization process is of critical importance for the Web stack because of its inherent complexity. \webcore starts from a (well-customized) general-purpose design and incorporates modest specialization for the most lucrative target.
\end{itemize}

\paragraph{Prospective} I discuss three key steps in generalizing the principles and improving the proposal.

\begin{itemize}
  \item \textsc{Easing Developers' Effort~~} Although developers must be empowered, they should not be over-burdened. A key limitation of the current \greenweb proposal is that it requires developers to manually annotate Web applications with QoS information. To make \greenweb more practical, I will explore the feasibility of automatically applying \greenweb annotations.
  
  The rationale for designing an automatic annotation system is twofold. First, some Web developers may not want to spend the extra effort of manual annotation, such as for legacy applications whose code logic might be obscure. Second, in complex Web applications with many DOM events, manually going through all events and identifying their QoS type and QoS target could be cumbersome. In both cases, an automated system would automatically find all the events and annotates them with the two QoS abstractions, enabling QoS-aware energy-efficiency optimizations without developers' intervention.
  
  I will also evaluate the trade-off between automatic and manual annotation. Specifically, I will use three metrics: coverage, accuracy, and effort. \textit{Coverage} is measured by the percentage of events that can be annotated. \textit{Accuracy} is assessed by whether an event's QoS type and QoS target information can be correctly annotated. \textit{Effort} is measured by the amount of time required to perform annotation. Intuitively, the manual annotation approach can achieve 100\% coverage at 100\% accuracy--after all, developers have all the knowledge of the exact application logic, theoretically. However, manual annotation would also lead to significant annotation effort, especially for production-scale Web applications. On the other hand, the automatic annotation approach would require much less annotation effort with the caveat of not being able to reach a perfect accuracy and coverage. My future work will quantify such a comparison and make a step further toward pushing \greenweb to practice.
 
  \item \textsc{Composability of GreenWeb~~} There are two aspects of composability that need to be addressed: program-level and abstraction-level. At the program level, an ideal \greenweb-annotated program can be integrated with other non-annotated code while ensuring functionality (correctness). After all, \greenweb extensions concern with QoS and traditional language constructs concern with functionality (correctness). This composability ensures a \textit{separation of concerns} between QoS and functionality (correctness) in Web programming.
  
  The program-level composability also allows developers to annotate applications that make use of close-sourced libraries. For example, even if an event callback is implemented using a third-party animation library API whose source code is not available, developers should still be able to specify the event's QoS type as ``continuous'' as long as the API is guaranteed to produce an animation. This capability allows QoS annotations to remain largely unchanged even when a library's actual implementation evolves over time.
  
  In my planned work toward thesis defense, I will evaluate the program-level composability as it is a short-term concern that affects the usability of \greenweb extensions. The key is to make sure that the  runtime implementation of QoS annotations is ``sandboxed'' and does not leak information to the ``functionality world.'' In this way, even when an QoS annotation is erroneous the application's functionality and correctness will not be interfered with.
  
  The other aspect of composability is at the QoS abstraction level. It is not clear if one can easily express a new QoS abstraction based on the two proposed QoS abstractions, QoS type and QoS target. The next step of Web language research should understand how to design QoS ``primitives,'' based on which complex, higher-level QoS abstractions can be easily composed. This composability is critical in the long term when we will see new user interaction forms and new ways that users assess QoS experience. Enumerating every single type of QoS in a programming model is not scalable. With composability at the QoS abstraction level, however, a Web programming model could simply provide a library that lets developers construct different QoS specifications in a completely customized way.
  
  I will leave the abstraction-level composability for the next phase of research because it is a long-term concern that involves extensively surveying future human-computer interaction forms and new QoS specifications. I believe that the QoS type and QoS target abstractions are sufficient for expressing predominant QoS specifications on today's mobile devices.
   
  \item \textsc{Self-adaptive Web Runtime~~} The Web is becoming a universal application development platform in the Internet-of-Things era. The underlying Web runtime is facing a unique challenge of operating on different device capabilities and in vastly different environments with different QoS, energy, and reliability constraints. The future Web runtime must be self-adaptive. As a first step, I will conduct two research items. First, I will evaluate the sensitivity of \webrt with respect to different QoS targets. For example, I will understand can the webpage-aware scheduler achieve the same level of energy savings when the cut-off latency is relaxed. Second, I will evaluate the sensitivity of \webrt on mobile devices with different capabilities. For instance, I will evaluate \webrt on an ARM Cortex A7 processor to mimic a smartwatch-level processor (as opposed to a smartphone).
\end{itemize}
